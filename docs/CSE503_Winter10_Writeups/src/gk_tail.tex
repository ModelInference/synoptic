\begin{figure}[!t]

\begin{lstlisting}[mathescape=true,morekeywords={Input,Output}]
Input: $(V,E)$
Let $S$ := k-equiv($G$)
While $S \neq \emptyset$
  Let ($s_1$, $s_2$) = DE.getMerge($(V,E)$, $S$)
  Let $(V',E')$ = k-merge($(V,E)$, $s_1$, $s_2$)
  If DE.checkGraphValidity($(G',V')$)
    $(V,E)$ := $(V',E')$
    $S$ := k-equiv($(V,E)$)
  Else
    $S$ := $S - \{(s_1, s_2)\}$
Output: $(V,E)$
\end{lstlisting}

\caption{The GK-Tail algorithm}
\label{fig:gk_alg}

\end{figure}

%%%%%%%%%%%%%%%%%%%%%%%%%%%%%%%%%%%%%%%%%%%%%%%%%%%%%%%%%%%%%%%%%%%%%
\section{The GK-Tail Algorithm}
\label{sec:gk_tail}
%%%%%%%%%%%%%%%%%%%%%%%%%%%%%%%%%%%%%%%%%%%%%%%%%%%%%%%%%%%%%%%%%%%%%

We now describe our adaptation of the GK-Tail algorithm. This
algorithm consists of two steps. First the algorithm merges all
input-equivalent traces, which are sequences of the same messages but
with potentially different fields. This is an initial data-reduction
step. Second, the algorithm starts with the most refined graph
representation of the trace and proceeds to coarsen this graph at each
step by merging some number of nodes. It terminates when it cannot
find any nodes to merge, either because they are no two nodes that are
similar enough, or because a merge of any two nodes would violate a
temporal trace invariant.

% \subsection{Partitioning the trace}

% Partitioning of the trace first merges input-equivalent traces. 

% ~\footnote{We plan to use Daikon to detect likely
%   invariants over these parameters in the future.}

% \item Use \texttt{Daikon}~\cite{Daikon} to come up with the most
%   general representations for the sets of values on the arcs. This
%   produces an initial FSM.

% Next the trace is partitioned into a set of traces using a partition
% function. 

\subsection{Coarsening}
\label{subsec:gk_tail_coarsening}

Let $G=(S,M)$ be the graph obtained by treating every trace as a
sequence of states with edges between the states representing
consecutive messages in the trace. We then define two helper
functions:

\begin{itemize}

\item \texttt{k-equiv(S,M)} returns $T \subseteq S \times S$ such that
  for all $(s,t)\in{T}$ we have that $s$ is $k$-similar to $t$ in
  $G$. By this we mean that the directed graph rooted at $s$, which is
  a subgraph of $G$ with maximum path length of $k$ is a subgraph of a
  similarly constructed graph rooted at $t$. There is a stronger
  version of $k$-similarity which mandates that the subgraphs be
  equivalent, however, we ignore this form in this paper.

\item \texttt{k-merge($G,s_1, s_2$)} returns $G'$ that is obtained
  from the graph $G$ by merging the nodes $s_1$ and $s_2$ up to depth
  $k$.

\end{itemize}

Now we can formulate our modified GK-Tail algorithm as shown in
Figure~\ref{fig:gk_alg}. The $k$ in this algorithm is a key parameter
that can be thought of as the degree of dependency between
messages. If messages are completely independent, a $k$ value of 1 is
sufficient. However, if a message influences a message downstream that
is $n$ messages away, $k$ should be set to a value $\geq n$. A smaller
$k$ value produces more compact representations, but it does so at the
loss of potential dependency information. A larger value of $k$
retains this information, however, it can be difficult to set the
right value of $k$. Moreover, a static value of $k$ might not be
appropriate for all steps of a protocol. At the moment our algorithm
uses a static value of $k$.

% \textbf{TODO: describe what we do: what value of k do we use? and
%   why?}

\subsection{Complexity of GK-Tail}

In the worst case, the GK-Tail algorithm will have to consider all
pairs of states in the current graph, if they are all $k$-equivalent,
before finding a pair that, when merged, gives a valid
representation. When a merge is performed, exactly one state is
removed from the graph. Thus, in the worst case the time-complexity of
the algorithm is $O(kn^3)$. Predictably, this does not scale well and
in practice does not handle input traces with more than 1000 entries.

\subsection{Divide and Conquer Optimization}

We address the poor scalability of GK-Tail algorithm by employing a
divide-and-conquer approach. We make the observation that the GK-Tail
algorithm may be applied independently to subgraphs of the original
graph as long as the shortest path connecting the subgraphs is no
shorter than $k$. We implement this by dividing the initial graph and
performing GK-Tail on each of the resulting sub-graphs before merging
the resulting graphs and performing GK-Tail on the merged graph. The
theoretical worst-case time-complexity is unchanged, but we find that
this divide and conquer approach works well in practice and makes the
algorithm scale effectively to traces of more than 20,000 entries (see
section~\ref{subsec:perf_eval} for more).
