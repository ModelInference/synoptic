%%%%%%%%%%%%%%%%%%%%%%%%%%%%%%%%%%%%%%%%%%%%%%%%%%%%%%%%%%%%%%%%%%%%%
\begin{abstract}
%%%%%%%%%%%%%%%%%%%%%%%%%%%%%%%%%%%%%%%%%%%%%%%%%%%%%%%%%%%%%%%%%%%%%

Distributed system implementations are difficult to debug and
understand. Often the only way to gain insight into a system is to
track messages sent between nodes, which may be implementing multiple
concurrently executing distributed protocols. There are few tools to
help with this task -- a common approach is to log all node events and
then manually inspect the logs for clues that might reveal why the
system behaves the way that it does.

To help with this task we developed \emph{Synoptic} -- a summarization
tool that takes the system's logged messages as input and outputs a
finite state automata representation. Compared to the raw message
traces, the resulting representation is visually concise, and is also
rich with inferred temporal relationships mined from the messages.

We describe how Synoptic works, and present its core algorithms and
the intuition that led to their design. We evaluate Synoptic by
applying it to a variety of real and synthetic distributed systems'
traces. We evaluate its performance with benchmarks, and we carry out
a user study with a distributed systems developer to evaluate its
utility. Our results suggest that Synoptic has reasonable overhead for
an offline analysis tool, and that it helps to augment distributed
system designers understanding of system behavior.

\end{abstract}

