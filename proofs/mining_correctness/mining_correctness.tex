\documentclass[a4paper,10pt]{scrartcl}
\usepackage[utf8x]{inputenc}

\usepackage{amsmath}

\usepackage{amsthm}
\usepackage{latexsym}
\usepackage{amsfonts}
\newtheorem{definition}{Definition}
\newtheorem{theorem}{Theorem}
\newtheorem{lemma}{Lemma}


%opening
\title{Mining Correctness}
\author{Sigurd Schneider\\sigurd@ps.uni-saarland.de}

\newcommand{\tr}{\ensuremath{l_0,...,l_n}}
\newcommand{\tri}[1]{\ensuremath{l_{#1},...,l_n}}
\newcommand{\NFBy}[2]{\ensuremath{\textup{{#1}\,\textsf{NFby}\,{#2}}}}
\newcommand{\AFBy}[2]{\ensuremath{\textup{{#1}\,\textsf{AFby}\,{#2}}}}
\newcommand{\AP}[2]{\ensuremath{\textup{{#1}\,\textsf{AP}\,{#2}}}}

\newcommand{\mine}{\ensuremath{\textup{\texttt{mine}}}}
\newcommand{\impl}{\ensuremath{\rightarrow}}
\newcommand{\Def}{\ensuremath{\textup{Def. }}}
\newcommand{\Lem}{\ensuremath{\textup{Lem. }}}
\newcommand{\until}{\ensuremath{\textup{\,\textsf{U}\,}}}

% \newcommand{\lor}{}

\begin{document}

\maketitle

\section{Setup}
In full generality, a trace is a possibly infinite sequence of states, where each state is associated with a set of labels. In this setting, 
a trace is restricted to be a finite sequences of labels, capturing that event traces are always finite, and exactly one event happens at a time. 

We consider LTL syntax as usual, but adopt the semantics to the restricted trace definition. Given a trace $\pi=\tr$, we denote by $\pi_i$ the 
sequence $\tri{i}$. 



\begin{definition}(LTL Semantics)
Let $\pi=\tr$ be a trace. Then the satisfiability relation $\models$ is defined as follows:
\begin{align*}
 \tr\models{l}&\iff l_0=l\\
 \tr\models{\Box{t}}&\iff\forall{i}\in[0,n]:{}\pi_i\models{t}\\
 \tr\models{\Diamond{t}}&\iff\exists{i}\in[0,n]:{}\pi_i\models{t}\\
 \tr\models{{t_1}\until{t_2}}&\iff\exists{i}\in[0,n]:{}\pi_i\models{t_2}\land\forall{j}\in[0,i-1]:\pi_j\models{}t_1
\end{align*}
\label{sem}
The propositional connectives $\land,\lor,\impl,\neg$ are lifted as usual.
\end{definition}

\begin{definition}
The shorthands \textsf{NFby}, \textsf{AFby}, \textsf{AP} for LTL formulas are defined for all labels $a,b$ as follows:
\begin{align*}
 \NFBy{a}{b}&:= \Box(a\rightarrow\Box(\neg{b}))\\
 \AFBy{a}{b}&:= \Box(a\rightarrow\Diamond{b})\\
 \AP{a}{b}&:= (\Diamond a)\rightarrow{\neg{b}U{a}}
\end{align*}
\label{inv}
\end{definition}
\section{Assumptions}
The implementation is not directly verified. Instead, the a set of assumptions is stated, from which the correctness follows. 
The implementation must be inspected and it must be verified that these assumptions indeed hold for the implementation to apply 
the proof.
The mining procedure $\mine$ yields a set of invariants for a set of traces.

\begin{definition}[Assumptions]
Let $T$ be a set of traces. Then it is assumed that
\begin{align*}
 \NFBy{a}{b}\in\mine(T)\iff\forall \pi\in{T}:\forall{i}\in[0,n]:l_i=a\impl\forall{}j\in[i,n]:l_j\not=b\\
 \AFBy{a}{b}\in\mine(T)\iff\forall \pi\in{T}:\forall{i}\in[0,n]:l_i=a\impl\exists{}j\in[i,n]:l_j=b\\
 \AP{a}{b}\in\mine(T)\iff\forall \pi\in{T}:\forall{i}\in[0,n]:l_i=b\impl\exists{}j\in[0,i]:l_j=b
\end{align*}
\label{ass}
\end{definition}

\section{Correctness Proof}
Correctness is soundness (every mined invariant holds), and completeness (all invariants that hold are mined). We establish these properties by prooving
the following theorem.
\begin{theorem}[Relative Correctness]
Let $T$ be a set of traces, and $a,b$ be labels.
\begin{align*}
 \NFBy{a}{b}\in\mine(T)&\iff\forall \pi\in{T}: \pi\models\NFBy{a}{b}\\
 \AFBy{a}{b}\in\mine(T)&\iff\forall \pi\in{T}: \pi\models\AFBy{a}{b}\\
 \AP{a}{b}\in\mine(T)&\iff\forall \pi\in{T}: \pi\models\AP{a}{b}
\end{align*}
\end{theorem}
\begin{proof}
To prove the first two statements, it suffices to expand the semantic definitions.
 \begin{align*}
     &\forall \pi\in{T}: \pi\models\NFBy{a}{b}&&\\
 \iff&\forall \pi\in{T}: \pi\models\Box(a\rightarrow\Box(\neg{b}))&&\Def\ref{inv}\\
 \iff&\forall \pi\in{T}: \forall{i\in[0,n]}:(\pi_i\models a)\rightarrow\forall{j\in[i,n]}:\pi_j\models\neg{b}&&\Def\ref{sem}\\
 \iff&\forall \pi\in{T}: \forall{i\in[0,n]}:l_i=a\rightarrow\forall{j\in[i,n]}:l_j\not={b}&&\Def\ref{sem}\\
 \iff&\NFBy{a}{b}\in\mine(T)&&\Def\ref{ass}
 \end{align*}

 \begin{align*}
     &\forall \pi\in{T}: \pi\models\AFBy{a}{b}&&\\
 \iff&\forall \pi\in{T}: \pi\models\Box(a\rightarrow\Diamond{b})&&\Def\ref{inv}\\
 \iff&\forall \pi\in{T}: \forall{i\in[0,n]}:(\pi_i\models a)\rightarrow\exists{j\in[i,n]}:\pi_j\models{b}&&\Def\ref{sem}\\
 \iff&\forall \pi\in{T}: \forall{i\in[0,n]}:l_i=a\rightarrow\exists{j\in[i,n]}:l_j={b}&&\Def\ref{sem}\\
 \iff&\AFBy{a}{b} \in\mine(T)&&\Def\ref{ass}
 \end{align*}
The third proof requires a lemma.
 \begin{align*}
     &\forall \pi\in{T}: \pi\models\AP{a}{b}&&\\
 \iff&\forall \pi\in{T}: \pi\models(\Diamond{b})\rightarrow(\neg{b})\until{a}&&\Def\ref{inv}\\
 \iff&\forall \pi\in{T}: (\exists{i\in[0,n]}:\pi_i\models b)\impl\exists{j\in[0,n]}:\pi_j\models{a}\land\forall{k\in[0,j-1]:\pi_k\models\neg{b}}&&\Def\ref{sem}\\
 \iff&\forall \pi\in{T}: (\exists{i\in[0,n]}:l_i=b)\impl\exists{j\in[0,n]}:l_j={a}\land\forall{k\in[0,j-1]:l_k\not={b}}&&\Def\ref{sem}\\
 \iff&\forall \pi\in{T}: (\forall{i\in[0,n]}:l_i\not=b)\lor(\exists{j\in[0,n]}:l_j={a}\land\forall{k\in[0,j-1]:l_k\not={b}})&&\Def\impl\\
 \iff&\forall \pi\in{T}: \forall{i\in[0,n]}:l_i=b\impl\exists{j\in[0,i]}:l_j={a}&&\Lem\ref{lem1}\\
 \iff&\AFBy{a}{b} \in\mine(T)&&\Def\ref{ass}
 \end{align*}

\end{proof}

\begin{lemma}
We proof that for all traces $\pi=\tr$, and for all labels $a,b$, we have
\begin{align*}
     &(\forall{i\in[0,n]}:l_i\not=b)\lor(\exists{j\in[0,n]}:l_j={a}\land\forall{k\in[0,j-1]:l_k\not={b}})\\
 \iff&\forall{i\in[0,n]}:l_i=b\impl\exists{j\in[0,i]}:l_j={a}
\end{align*}

 \label{lem1}
\end{lemma}
\begin{proof} Each direction is proven separately.
 \begin{itemize}
  \item[$\Rightarrow$] 
   The claim must follow from each of the disjuncts separately.
   \begin{itemize}
    \item Then $\forall{i\in[0,n]}:l_i\not=b$, and the claim is trivial.
    \item Then there is ${j\in[0,n]}$ such that $l_j={a}\land\forall{k\in[0,j-1]:l_k\not={b}}$. Let $i\in[0,n]$. The claim is proven by case analysis on $i$. 
    \begin{itemize}
     \item Case $i<j$. Then $\forall{k\in[0,j-1]:l_k\not={b}}$ and the claim holds trivially. 
     \item Case $i\geq{}j$. Then the claim holds since $l_j=a$.
    \end{itemize}
   \end{itemize}
   \item[$\Leftarrow$]
   We either have that $\forall{i\in[0,n]}:l_i\not=b$, in which case the claim holds by assumption, or there is an ${i\in[0,n]}$ such that $l_i=b$, and we 
   have to show the conjunction $\exists{j\in[0,n]}:l_j={a}\land\forall{k\in[0,j-1]:l_k\not={b}}$.

   Using $l_i=b$, the assumption provides $\exists{j\in[0,i]}:l_j={a}$, which implies the left conjunct that had to be shown. Now let $j$ be the smallest number such 
   that $l_j=a$. i.e. such that $\forall{j'}\in[0,j-1]:l_{j'}\not=a$. Consider the contra-positive of
   the assumption: $\forall{i\in[0,n]}:(\forall{j\in[0,i]}:l_j\not={a})\impl{}l_i\not=b$. Since $j$ is the smallest witness it follows that 
   $\forall{k\in[0,j-1]}:l_k\not=b$, which is the right conjunct and thus completes the proof.
 \end{itemize}
\end{proof}



\end{document}
